\documentclass[a4paper,11pt,DIV=calc,tablecaptionabove,headinclude,twoside]{article}
\usepackage[utf8]{inputenc}
\usepackage{ngerman}
\usepackage{amsmath}
\usepackage{amssymb}
\usepackage{array}
\usepackage{booktabs}
\usepackage{fancyhdr}                   %% Seite, Kopf- und Fußzeile anpassen (siehe \usepackage{geometry})
\usepackage{flafter}                    %% verhindert dass Abbildungen vor Überschrift kommen
\usepackage{float}
\usepackage[pdfpagelabels=true]{hyperref}
\usepackage{geometry}                   %% Seitenmaße verändern (siehe \usepackage{fancyhdr}) !!! Das Paket IMMER nach hyperref einbinden!!!
\usepackage{graphicx}                   %% Einbinden von Bildern mit \includegraphics[]{×}
\usepackage{tabularx}
\geometry{a4paper, top=27mm, left=30mm, right=20mm, bottom=30mm, headsep=10mm, footskip=10mm}
\usepackage[decimalsymbol=comma, binary-units=true, loctolang={DE:ngerman}]{siunitx}
% --------------------------- BibTeX ---------------------------------------
\usepackage[square]{natbib}                 
%%%\bibliographystyle{natdin}           %% Referenzierung: Autor und Jahreszahl
%\bibliographystyle{lex2018}

% Title Page
\title{Experimentplanung LEX 2018}%:\\Messseeelefantenantennenbenennungsverfahren}
\author{Autoren}

\begin{document}
\maketitle



\section{Einleitung}
\subsection{Motivation}
Warum ist das Thema interessant bzw. relevant? Warum lohnt es sich, sich mit dem Thema zu beschäftigen? (0.5--1 Seiten)\\\\
Die Grenzschicht ist der Bereich der Atmosphäre, der direkt vom Boden beeinflusst ist. Dazu zählen bespielsweise der Einfluss von Reibung am Erdboden, Strahlungsprozesse und der generelle Energieaustausch zwischen Boden und Atmosphäre. Die meisten für uns bedeutsamen Wetterparameter, wie bodennahe Temperatur und Wind, die relativen Feuchte oder die Stabilität und Gewitterwahrscheinlichkeit hängen vom Zustand der Grenzschicht ab. Auch wichtige Wetterphänomene, wie die Nebelbildung oder die Land-Seewind Zirkulation finden hier statt.\\
Ein gutes Verständnis und eine präzise Modellierung der Grenzschicht ist jedoch nicht nur für eine gute Wettervorhersage unablässlich, sondern auch beim Bau großer Gebäude, dem Flugverkehr oder für die Optimierung von Windkraftanlagen von Bedeutung.\\
Im Rahmen der LEX bietet sich uns eine einzigartige Möglichkeit, den Aufbau der Grenzschicht in küstennahen Regionen besser zu verstehen, da wir kontinuierliche, zeitlich und vertikal hochaufgelöste in-situ Profilmessungen durchführen können. Von einigen Parametern, wie der Temperatur oder der Feuchte gibt es nur vertikal schlecht aufgelöste Profilmessungen. Durch unsere gute Auflösung selbst in höheren Bereichen der Grenzschicht erhoffen wir uns die Möglichkeit, die zeitliche Entwicklung der Grenzschicht im Messzeitraum besser zu verstehen und möglicherweise sogar neue Erkenntnisse aus diesen abzuleiten.
%\begin{itemize}
	
%	\item Grenzschicht ist die untere Schicht der Atmosphäre, die direkt vom Boden beeinflusst
%	wird
%	\item Reibung, Strahlungsprozesse, Energieaustausch zwischen Boden und Atmosphäre
%	Warum ist die Grenzschicht wichtig? Warum wollen wir sie modellieren?
%	\item  Lebensraum des Menschen
%	\item viele für uns wichtige Wetterparameter hängen vom Zustand der Grenzschicht ab
%	(Temperatur, Feuchte, Nebel, Wind, Land-Seewind Zirkulation, Stabilität)
%	- ausgeprägte tageszeitliche Schwankungen dieser Parameter
%	- wichtig für gute Wettervorhersage
%	- Windkraftanlagen
%	
%	Schwierigkeiten bei der Modellierung/
%	Was fehlt uns zum Verständnis der Grenzschicht?
%	- Turbulenz
%	- zeitliche Entwicklung der Grenzschicht
%	- Mischprofile Land/See (wichtig für Modelle, Gitterpunkte mit Land und See)
%\end{itemize}
%
%
%Kontinuierliche zeitlich und vertikal hochaufgelöste in-situ Messungen der
%atmosphärischen Grenzschicht bis 1000m Höhe existieren bislang nur sporadisch.

\subsection{Wissenschaftliche Ziele/Fragen}
Jedes Ziel mit Bullet-Punkten kurz und prägnant definieren; ein bis zwei -- auch stichwortartige -- Sätze genügen; nach Möglichkeit Hypothesen/Fragen formulieren, die im Experiment überprüft/beantwortet werden sollen. (wenige, 1 bis 5 Ziele)

\begin{itemize}
\item Entwicklung der 'Hardware', um Profilmessungen der Grenzschicht mithilfe eines Fesselballons oder eines Drachens von Temperatur und Feuchte (mit möglichst hoher vertikaler Auflösung) durchführen zu können.
\item Welche Sensoren eignen sich? Interne Variabilität, absolute Genauigkeit
\item Grenzen der Messmethoden prüfen: Wie geeignet ist der Fesselballon/Drachen/Drohne für welche Zwecke: ZIEL: Wie kann man die Messungen beider (aller drei) Messmethoden ideal kombinieren, um möglichst viele Informationen zu erhalten. Vertikalauflösung und zeitliche Auflösung der Arduinos, was sind die Möglichkeiten der Messung durch die Drohne, was sind die Grenzen?
\item Wie sehr beeinflusst der Downwash der Drohne die Messungen?
\item Drohne: Messungen über dem Meer: \textbf{Unterschied Grenzschicht Land Meer}, Entwicklung der Grenzschicht über Land, wie weit reicht der Einfluss des Meeres/Landes
\item Zeitskalen: Wie schnell entwickeln sich die neue Grenzschicht? Z.B., wenn sich die Anströmrichtung von Land auf Meer wechselt
\item Entwicklung und Test einer eigenen Messhardware zur Profilmessung von
    Temperatur, Feuchte und Druck
\item Untersuchungen zur Entwicklung der Grenzschicht (Tagesgang)
\item Wie beeinflusst die Anströmrichtung die Profile über der Insel? Ist ein
    Landseewind erkennbar?
\end{itemize}

\section{Stand der Forschung}
Was ist zum Thema bekannt? Welche Vorarbeiten (Paper, alte LEX-Berichte usw.) gibt es, an die angeknüpft werden kann?
- Übergang Tag-/Nachtgrenzschicht (Theresa)
- Übergang Land-/Seegrenzschicht (Henning)
- Drohnenmessungen (Jakob)
- Allgemeine Grenzschichtmessmethoden (Laura)
- Gibt es Grenzschichtmessungen mit einem ähnlich hohen Ballon? (Simi)

Morning transition case between the land and the sea breeze regimes Jimenez et
al
DOI: 10.1016/j.atmosres.2015.12.019

\section{Messstrategie}
Wie sieht die allgemeine Messstrategie aus? Welche Größen sollen z.~B. mit welcher zeitlichen Auflösung gemessen werden, um die wissenschaftliche Fragestellung beantworten zu können?
\subsection{Experimenteller Aufbau}
Welche Instrumente sollen wie (Messhöhe, Standortwahl, horizontale Abstände, ... ) aufgebaut werden? (nach Möglichkeit Skizzen und Pläne hinzufügen)
- 1km Messhöhe
- im Idealfall (Ballonschnur vertikal): 2, 20, 50, 100, 280, 460, 640, 820,1000 #TODO: nochmal nachdenken, Skizze machen
- Standort: Welcher Ort ist genehmigt? -> Hubschrauberlandeplatz (ca. 70m vond er Küste entfernt, keine Bäume)

- Welche Instrumente?: Temperatur, Feuchte, Druck an Arduino

- Drohne: Sensor unterhalb der Drohne (ca. 20cm) befestigt. Es sollen 2D profile als Ergänzung zum Ballon geflogen werden #TODO: Wie genau soll mit der Drohne geflogen werden?

- Arduinos senden Daten live: Aufbau: Slaves/Master
\subsection{Bedingungen/Erfordernisse}
Welche externen Bedingungen sind notwendig (Wetter)? Welche internen Erfordernisse sind notwendig (Betriebsmittel, Personal, externe Daten zur Durchführung usw.)? Sind Messdaten der anderen Gruppen erforderlich (Messstrategie absprechen)?


\begin{itemize}
\item externe Bedingungen:
	\item kein Niederschlag, auch keine Niederschlagwahrscheinlichkeit!
	\item max. Wind für Ballon #TODO Felix fragen
	\item max. Wind für Drohne: theoretisch 15m/s, wir schaffen es nur bis geringe Windgeschwindigkeit (max. 5m/s)
	\item für Land-Seewind: möglichst wenig Wind\\
	
\item interne Erfordernisse:
	\item Kine anderen Messungen erforderlich
	\item Helikite
	\item Drohne
	\item Flugerlaubnis
\end{itemize}

\subsection{Durchführung}
Beschreibung des regulären Messablaufs. Sind Intensivmessphasen geplant? Auf welche besonderen Ereignisse soll wie reagiert werden? 
Wann werden zusätzliche Messungen (z.\,B. Radiosondenaufstiege) durchgeführt?
Wie sollen die Messdaten aufgezeichnet werden? 

\textbf{Reguläre Durchführung:}\\
\item Ab Sonnenaufgang (06:15, 28.08; 06:35, 08.09) muss gemessen werden. Ballon schon vorher aufgebaut werden: Tag 1: 2h für Aufbau planen, alle. Anschließend anpassen (kürzere Vorbereitungszeit, evtl. weniger Leute, je nachdem, wie schnell wir sind.
\item Messzeitraum: Sonnenaufgang bis 15Uhr. Dann soll der Ballon unten sein.
\item durchgehende Messungen von T, Rh, p. In Auflösung: sekündlich.
\item Daten werden in Echtzeit an Master gesendet, können während des Messzeitraums ausgewertet werden.

\textbf{Zusätzliche Drohnen-Messungen :}\\
Es sind Intensivmessphasen mit der Drohne geplant.
\item Zwei Drohnenmessungen täglich, wenn die Grenzschicht schon aufgebaut ist.
\item Winkel zum Wind: parallel zum Wind)
\item Welche Form soll die Drohne fliegen?: Einmal 250m raus aufs Meer, einmal zurück in anderer Höhe und auf dieser Höhe weiter übers Land (insg. 500m), in neuer Höhe zurück (250m). Falls Akku leer, hier Messstop. Sonst nochmal über Meer in dieser Höhe (insg. weitere 500m). (Also Stufenprofil)
\item Evtl. Zusatzmessungen bei besonderen Ereignissen.
\subsection{Auswertung}
Wie sollen die Daten ausgewertet werden? Sind zusätzliche Daten (Satellitendaten, Modelldaten usw.) notwendig? Welche Grafiken oder Auswertungen sollen im Idealfall am Ende die wissenschaftlichen Fragen beantworten?

\subsection{Risiken}
Woran könnten Teile des Experiments scheitern? Welche Alternativen gibt es in diesen Situationen? (für jedes Risiko einen Absatz)
-Was tun bei Regen? LIDAR??
\section{Kosten}
Kosten für zusätzliches Material, Geräte, Betriebsmittel ...

\section*{Literatur}

\end{document}